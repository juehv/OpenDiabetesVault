\begin{changemargin}{4em}{4em} 

The OpenDiabetesVault-Plot trys to combine all supported datatypes in one comprehensive plot.
This enables fast data analysis for both, checking novel algorithms and analysing the health status of a patient.

The plot is divided into 3 zones:
\begin{itemize}
 \item Symbol bars: The upper symbol bar indicates technical information about the insulin pump and the glucose sensor systems.
 The second symbol bar is indicating abstracted locations of the patients.
 The third symbol bar indicates physical activity in three intense steps.
 The fourth and lowest symbol bar shows the stress level (currently calculated from heart variability values).
 \item Continuous data: The middle zone shows time series data.
 Continuous glucose measurement data (or flash glucose measurement data), heart rate data and machine learning prediction results are shown as line plot.
 Blood glucose values are shown as dots and are connected with a line plot for better readability.
 Blood glucose values taken for calibration or bolus calculation are specially marked by a symbol within the plot.
 The sleeping phases are indicated by the background color.
 \item Discrete data: This zone shows the amount of insulin and meals logged by the insulin pump. 
 Bolus insulin and meals are shown as bar plot.
 Meals are taken from the bolus insulin calculation.
 Therefore, patients should always put the exact estimated amount of carbs in it. 
 If some insulin savings are intended, patents should modify the calculation result of the bolus calculator.
 Basal insulin is shown as stepped line plot, which is a mixture of the bolus profile and temporary modification.
 It is not shown if it is a temporary basal rate or a profile rate, but it the plot shows the actual injected basal insulin.
 However, if the pump can act autonomously, a different colored background indicates this action.
\end{itemize}

\end{changemargin}
